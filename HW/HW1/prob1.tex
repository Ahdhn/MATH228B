\section{Problem No.1}
\subsection{Problem Description:} 
Consider the advection equation
$$
u_{t}+ au_{x}=0
$$
on the interval $[0,1)$ with periodic boundary conditions. Space is discretized as $x_{j}=j\Delta x$ for $j=0...N-1$, so that $\Delta x=\frac{1}{N}$. Descretize the spatial derivative with the second-order centered difference operator.

\begin{enumerate}
\item For simplicity, assume $N$ is odd. The eigenvectors of the centered difference operator are
$$
v_{j}^{k} = exp(2\pi jkx_{j})
$$
for $k=\frac{-(N-1)}{2}...\frac{(N-1)}{2}$. Compute the eigenvalues.
\item Derive a time step restriction on a method-of-lines approach which uses classical fourth-order Runge-Kutta for time stepping.
\end{enumerate}

\subsection{Solution:}\label{sec:prob1_sol}
\paragraph{Eigenvalues:}
The periodic boundary condition is expressed as 
$$
u(0,t) = u (1,t) 
$$
which physically means that whatever flows out at the outflow boundary flows back in at the inflow boundary (assuming $a>0$)\cite{hoffmann2000computational}. Thus we have $u_0(t)=u_{n-1}(t)$, which contributes as a single unknown($u_{n-1}(t)$). Using the centered difference operator for spatial derivative, we get 
$$
u_{j}^{\prime}(t) = \frac{-a}{2\Delta x}(u_{j+1}(t)-u_{j-1}(t))
$$
\vspace{2mm}

For $j=1\longrightarrow u_{1}^{\prime}(t)=\frac{-a}{2\Delta x}(u_{2}(t)-u_{0}(t))$ 
\vspace{2mm}

or $j=1\longrightarrow u_{1}^{\prime}(t)=\frac{-a}{2\Delta x}(u_{2}(t)-u_{n-1}(t))$ 
\vspace{2mm}

For $j=2\longrightarrow u_{2}^{\prime}(t)=\frac{-a}{2\Delta x}(u_{3}(t)-u_{1}(t))$ 
\vspace{2mm}

And so forth, till we reach $i=n-1$

For $i=n\longrightarrow u_{n-1}^{\prime}(t)=\frac{-a}{2\Delta x}(u_{1}(t)-u_{n-2}(t))$ 
\vspace{2mm}

The system of linear equations can be written in the following form

\[
\left| 
\begin{array}{c}
u_{1}^{\prime}(t) \\
u_{2}^{\prime}(t) \\
u_{3}^{\prime}(t) \\
...\\
...\\
...\\
u_{n-2}^{\prime}(t) \\
u_{n-1}^{\prime}(t) \\
\end{array} 
\right|
=
\left| 
\begin{array}{ccc ccccc}
0       & \alpha  & 0      & 0      & ... & ... & 0 & -\alpha \\
-\alpha & 0       & \alpha & 0      & ... & ... & 0 & 0 \\
0       & -\alpha & 0      & \alpha & ... & ... & 0 & 0\\
... & ... & ... & ... & ... & ... & ... & ...\\
... & ... & ... & ... & ... & ... & ... & ...\\
... & ... & ... & ... & ... & ... & ... & ...\\
0 & 0 & ... & ... & 0 & -\alpha & 0       & \alpha\\
\alpha & 0 & ... & ... & 0 & 0       & -\alpha & 0\\
\end{array} 
\right|
\left| 
\begin{array}{c}
u_{1}(t) \\
u_{2}(t) \\
u_{3}(t) \\
...\\
...\\
...\\
u_{n-2}(t) \\
u_{n-1}(t) \\
\end{array} 
\right|
\] 

where $\alpha = \frac{-a}{2\Delta x}$. We notice that A is \emph{skew-symmetric} since $A_{T}=-A$, and thus its eigenvalues are pure imaginary \cite{leveque2007finite}. Additionally, the nonzero eigenvalues come in pairs, each the negation of the other. 
The eigenvalues of the matrix can be derived by assuming that $u_{j}(t)$ take the form $e^{i\zeta x_{j}}$, where $\zeta$ is the wave number and $i=\sqrt{-1}$. By substituting into the discretized equation, it becomes 
$$
u_{j}^{\prime}(t) = \frac{-a}{2\Delta x}(e^{i\zeta (x_{j}+\Delta x)}-e^{i\zeta (x_{j}-\Delta x)}) = \frac{-a}{2\Delta x}e^{i\zeta x_{j}}(e^{i\zeta \Delta x}-e^{-i\zeta \Delta x})
\\
u_{j}^{\prime}(t) = \frac{-a}{2\Delta x}e^{i\zeta x_{j}}2isin(\zeta \Delta x)=\frac{-a}{\Delta x}e^{i\zeta x_{j}}isin(\zeta \Delta x)
$$
The wave number can be written as $\zeta=2\pi k$. Thus,
$$
u_{j}^{\prime}(t) = \frac{-ia}{\Delta x}sin(2\pi k \Delta x)u_{j}(t)
$$
Thus, the eigenvalues of the matrix are
$$
\lambda_{k} = -\frac{ia}{\Delta x}sin(2\pi k \Delta x), \;\; k=1,2,.... n-1
$$


\paragraph{Stability Restrictions:}
To derive the stability restriction, we start by expressing the advection equation as $u_{\prime}(t)=\lambda u(t)$, where $\lambda$ is the eigenvalue. After applying the descritization method (fourth-order Runge-Kutta), we obtain $U^{n+1}=R(z)U^{n}$, such that $R(z)$ is some function of $z=\lambda\Delta t$. $R(z)$ is a polynimal for explicit methods and rational for implicit methods \cite{leveque2007finite}. Then, the range of absolute stability will be 

$\mathcal{S}={z\in \mathbb{C} : |R(z)|\leq 1}$

We start with applying the fourth-order Runge-Kutta method on the advection equation. We obtain
$$
U^{n+1} = U^{n} + \frac{\Delta t}{6}[f(Y_{1},t_{n})+2f(Y_{2},t_{n}+\frac{\Delta t}{2})+2f(Y_{3},t_{n}+\frac{\Delta t}{2})+f(Y_{4},t_{n}+\Delta t)]
$$
such that 
$$
Y_{1}=U^{n}
\\
Y_{2}=U^{n}+\frac{1}{2}\Delta t f(Y_{1},t_{n})
\\
Y_{3}=U^{n}+\frac{1}{2}\Delta t f(Y_{2},t_{n}+\frac{\Delta t}{2})
\\
Y_{4}=U^{n}+\Delta t f(Y_{3},t_{n}+\frac{\Delta t}{2})
$$
The full derivation can be found in the Appendix. We obtain at the end 
$$
U^{n+1} = U^{n}[1+z+\frac{1}{2}z^{2}+\frac{1}{6}z^{3}+\frac{1}{24}z^{4}]
$$ 

From the definition of absolute stability introduced earlier, we have the following restriction 
$$
|1+z+\frac{1}{2}z^{2}+\frac{1}{6}z^{3}+\frac{1}{24}z^{4}| \leq 1
$$
The above equation should give the restriction over the time step by substituing $z=\lambda \Delta t$. 
